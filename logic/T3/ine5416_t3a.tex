\documentclass{article}

\usepackage[utf8]{inputenc}
\usepackage[a4paper, left=20mm, right=20mm, top=20mm, bottom=20mm]{geometry}

\begin{document}

\subsubsection*{INE5416 - Paradigmas de Programação (2015/2) \\
    Gustavo Zambonin, Lucas Kramer de Sousa,
    Marcello da Silva Klingelfus Junior \\
    Programação Lógica - T3A
}

\textbf{Nota}: todos os excertos de código foram executados com
\texttt{swipl -qs ine5416\_t3a.pl} e chamados no interpretador. Recomenda-se,
também, utilizar aspas duplas em nomes de arquivos como argumentos.

\begin{itemize}

\item \texttt{negative}

A função atribuída ao predicado consiste em inverter a intensidade dos
\textit{pixels} de uma dada imagem no formato PGM. Utilizando predicados
fornecidos pelo arquivo \texttt{img.pl}, o arquivo é lido para uma matriz
onde cada elemento representa a intensidade de preto, variando de 0 a 255,
de um \textit{pixel}, junto com suas coordenadas. Iterando sobre esta lista
de tuplas e aplicando a operação de inversão de intensidade ($255 - I$) em
um predicado reservado para esta função (\texttt{calc\_neg}), os resultados
são então salvos em uma nova lista e escritos em um novo arquivo pelo
predicado \texttt{output}, utilizando um sufixo adequado. A saída do programa
estará na mesma pasta deste.
\begin{verbatim}
?- negative("ufsc.pgm").
Output: ufsc_neg.pgm
true.
\end{verbatim}

\item \texttt{mean}

Designa-se a este predicado a responsabilidade de calcular a média da
intensidade entre cada dupla de \textit{pixels} de duas imagens. O funcionamento
é similar ao descrito acima, porém é necessário ler duas imagens e repassar as
suas respectivas matrizes para um predicado auxiliar (\texttt{calc\_am}), que
realizará o procedimento aritmético e retornará a nova matriz, para ser escrita
por \texttt{output}. A saída terá sempre o mesmo nome de arquivo. O predicado
deverá retornar \texttt{false.} se as imagens não tiverem o mesmo tamanho.
\begin{verbatim}
?- mean("ufsc.pgm", "ufsc_neg.pgm").
Output: avg.pgm
true.
\end{verbatim}

\item \texttt{isolated\_pixel}

Neste caso, é preciso procurar, em todos os \textit{pixels} da imagem, os que
têm intensidade maior do que seus diretamente adjacentes (cima, baixo,
esquerda e direita). Utiliza-se o predicado \texttt{n4}, previamente fornecido,
para obter uma lista com estes adjacentes e comparar numericamente suas
intensidades. O processo pode se tornar indesejavelmente lento para imagens
de grandes dimensões. A imagem original `\texttt{ufsc.pgm}' foi modificada para
a adição de alguns \textit{pixels} isolados.
\begin{verbatim}
?- isolated_pixel("ufsc.pgm").
[ (1,1,50), (1,4,50), (1,16,200)]
true.
\end{verbatim}

\item \texttt{path\_between}

Um caminho válido entre dois \textit{pixels} dá-se quando todas as intensidades
neste caminho, considerando também os adjacentes, são iguais ou maiores do que
a intensidade da origem. Para obter este resultado, verifica-se passo a passo
possíveis caminhos entre o \textit{pixel} origem e o destino, passados como
argumento para o predicado, e se estes respeitam a regra original.
\begin{verbatim}
?- path_between("ufsc.pgm", (0, 0, 0), (1, 1, 0), A).
Path:
A = [ (0, 0), (0, 1), (1, 1)].
\end{verbatim}

\end{itemize}

\end{document}
