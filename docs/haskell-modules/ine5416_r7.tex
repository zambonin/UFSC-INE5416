\documentclass{../sftex/sftex}

\usepackage{amsfonts, mathtools}

\title{Módulos em Haskell}
\author{Gustavo Zambonin}
\email{gustavo.zambonin@grad.ufsc.br}
\src{https://github.com/zambonin/ufsc-ine5416}
\uniclass{Paradigmas de Programação}
\classcode{UFSC-INE5416}

\begin{document}

\maketitle

\textbf{Nota}: todos os excertos de código foram executados com
\verb!ghci ine5416_r7.hs! e chamados no interpretador. \\ \\
Utilizando o conceito de módulos, foi criado um programa que
permite calcular valores para as funções hiperbólicas listadas abaixo,
sendo $e = \sum\limits_{n=0}^\infty \frac{1}{n!} = \frac{1}{0!} +
\frac{1}{1!} + \frac{1}{2!} + \frac{1}{3!} + \dots$ a base do logaritmo
natural, calculada pela soma dos 1000 primeiros termos da série.
\begin{itemize}
    \item $\sinh x = \dfrac{1 - e^{-2 \cdot x}}{2e^{-x}}$
    \item $\cosh x = \dfrac{1 + e^{-2 \cdot x}}{2e^{-x}}$
    \item $\tanh x = \dfrac{\sinh x}{\cosh x}$
    \item $\coth x = \dfrac{\cosh x}{\sinh x}$
\end{itemize}
O módulo retorna, comparativamente à funções nativas da linguagem Haskell,
valores significativos apenas até a sexta casa decimal, por conta do
número de computações limitadas da constante $e$, como pode ser
visto abaixo.

\begin{verbatim}
*Hyperbolic> 1/tanh 1
1.3130352854993315
*Hyperbolic> value(Coth 1)
1.3130355
\end{verbatim}
Outras funções podem ser chamadas, respectivamente, por
\verb!value(Sinh x)!, \verb!value(Cosh x)! e \verb!value(Tanh x)!,
para $x \in \mathbb{R}$.

\end{document}
